% Created 2017-11-14 Tue 13:10
% Intended LaTeX compiler: pdflatex
\documentclass[11pt]{article}
\usepackage[utf8]{inputenc}
\usepackage[T1]{fontenc}
\usepackage{graphicx}
\usepackage{grffile}
\usepackage{longtable}
\usepackage{wrapfig}
\usepackage{rotating}
\usepackage[normalem]{ulem}
\usepackage{amsmath}
\usepackage{textcomp}
\usepackage{amssymb}
\usepackage{capt-of}
\usepackage{hyperref}
\author{Charles Martin}
\date{\today}
\title{Hello World and Why It's Important}
\hypersetup{
 pdfauthor={Charles Martin},
 pdftitle={Hello World and Why It's Important},
 pdfkeywords={},
 pdfsubject={},
 pdfcreator={Emacs 25.3.1 (Org mode 9.1.2)}, 
 pdflang={English}}
\begin{document}

\maketitle
\tableofcontents


\section{Introduction}
\label{sec:orgb9f4cbf}

One of the problems that faces beginning programmer -- and every
programmer ever when they start a new project -- is what to do with
that new blank project directory.  The traditional way to deal with that is to build a
simple program that demonstrates you can indeed write some simple
program that runs.

Since (at least) the original version of Kernighan and Ritchie's [[The
C Programming Language][]] that first simple program is called "Hello,
world!". In C, the hello, world program is:

\begin{verbatim}
#include <stdio.h>

int main(int argc, char** argv){
    printf("Hello, world!\n");
    return 0;
}
\end{verbatim}

Compiling and running it, simplest case is like this:

\begin{verbatim}
$ cc hello.c
$ ./a.out
Hello, world!
$
\end{verbatim}

In Java, it looks like:
\begin{verbatim}
/**
 * Canonical hello world in Java. Note this *must* be in a file named
 * 'Hello.java'
 */
public class Hello {
    public static void main(String[] argv){
        System.out.println("Hello, world!");
    }
}

\end{verbatim}
\begin{verbatim}
$ javac Hello.java
$ java Hello
Hello, world!
$
\end{verbatim}

In C++ it's:
\begin{verbatim}

#include <iostream>

using namespace std;

int main(int argc, char ** argv){
    cout << "Hello, world!" << endl;
    return 0;
}

\end{verbatim}

And sure enough:
\begin{verbatim}
$ g++ hello.cpp
$ ./a.out
Hello, world!
$
\end{verbatim}

That's enough examples for now. If you want to see "Hello, world!" in
every computer language you can name, there's an online repository of
"hello, world" programs at \href{https://helloworldcollection.github.io/}{The Hello World Collection}.

This is obviously not a very interesting program in itself, but
writing it and running it means that you must:

\begin{itemize}
\item find an editor or IDE that you can use for the particular language;
\item successfully write the code and save it;
\item compile and run, or invoke the interpreter to run the program.
\end{itemize}

These are, not coincidentally, all the problems you must solve to
start writing a program in any language or environment. This is even
true in an old-fashioned punch-card environment -- you needed to punch
cards more or less like this:

\begin{verbatim}
//FORT     EXEC PGM=IEYFORT,REGION=100K                                 00020000
//SYSPRINT DD  SYSOUT=A                                                 00040000
//SYSPUNCH DD  SYSOUT=B                                                 00060000
//SYSLIN   DD  DSNAME=&LOADSET,DISP=(MOD,PASS),UNIT=SYSSQ,             X00080000
//             SPACE=(80,(200,100),RLSE),DCB=BLKSIZE=80                 00100014
//LKED EXEC PGM=IEWL,REGION=96K,PARM=(XREF,LET,LIST),COND=(4,LT,FORT)   00120000
//SYSLIB   DD  DSNAME=SYS1.FORTLIB,DISP=SHR                             00140000
//SYSLMOD  DD  DSNAME=&GOSET(MAIN),DISP=(NEW,PASS),UNIT=SYSDA,         X00160000
//             SPACE=(1024,(20,10,1),RLSE),DCB=BLKSIZE=1024             00180014
//SYSPRINT DD  SYSOUT=A                                                 00200000
//SYSUT1   DD  UNIT=SYSDA,SPACE=(1024,(100,10),RLSE),DCB=BLKSIZE=1024, X00210018
//             DSNAME=&SYSUT1                                           00220018
//SYSLIN   DD  DSNAME=&LOADSET,DISP=(OLD,DELETE)                        00240000
//         DD  DDNAME=SYSIN                                             00260000
//GO EXEC PGM=*.LKED.SYSLMOD,COND=((4,LT,FORT),(4,LT,LKED))             00280000
//FT05F001 DD  DDNAME=SYSIN                                             00300000
//FT06F001 DD  SYSOUT=A                                                 00320000
//FT07F001 DD  SYSOUT=B            
C     Hello World in Fortran

      PROGRAM HELLO
      WRITE (*,100)
      STOP
  100 FORMAT (' Hello World! ' /)
      END
/*
\end{verbatim}

(Warning, don't take that JCL -- the stuff at the top -- very
seriously. It's an example JCL that will compile and run a FORTRAN
program, but it's utterly untested.)

Now, this looks pretty simple -- super-simple, frankly. There are a
lot of potential complexities you could run into -- and we'll talk
about them more in the next section -- but even this simple a "Hello,
world!" can expose surprises.

For example, my original version of the C++ program was:

\begin{verbatim}
#include <iostream.h>

int main(int argc, char ** argv){
    cout << "Hello, world!" << endl;
    return 0;
}
\end{verbatim}

The results of compiling it were:
\begin{verbatim}
541 $ g++ hello.cpp
hello.cpp:1:10: fatal error: 'iostream.h' file not found
#include <iostream.h>
         ^~~~~~~~~~~~
1 error generated.
\end{verbatim}

"What?" I wondered. (Actually I used three words, but the meaning was
the same.) Now, I haven't written much C++ in a long time, but it
looked reasonable.  But C++ has gone through a lot of modifications
since I was using it in the late 90's, and I'd missed the change from
`<iostream.h>` to `<iostream>`.

In this case, it wouldn't have been a really major issue even if I'd
written lots of code before trying an initial compile (but don't do
that! More on this in another note.) On the other hand, making this
program compile and run both exhibited the problem and reminded me to
grab my copy of Stroustrup's \href{https://www.amazon.com/C-Programming-Language-4th/dp/0321563840}{The C++ Programming Language} and refresh
myself before I try anything too radical.

\section{Beyond Hello, World}
\label{sec:org413e75d}

So far, we've looked at some trivial "Hello, world" sorts of
programs. While they \emph{are} trivial, they're not useless -- every one
of them means you've solved whatever problems there are before you can
start writing a program.

But sometimes it's not that simple.  In the Java world, the "hello,
world" above demonstrates that you've successfully installed both the
Java runtime and a Java Development Kit (or installed an IDE for Java,
like Net Beans.) 

If you intend to write a significant program in Java, you have to
choose a build tool and set up the environment it requires, or you
have to become enough a wizard in that build tool to make up your own
set of conventions (pro tip: don't.)

Other JVM languages have similar problems because of some design
decisions made early in the life of Java that turn out to be perhaps
less wise than they seemed. (Which ones, you ask? Some of them: tying
the names of classes to their file names, tying packaging to the
directory structure, which makes the directory layout be reflected in
the source code: move a file or rename a package, and you break all
the code associated with that. But that's another article.)

\subsection{Making a Java Hello, World project using Maven}
\label{sec:org16d2471}

The most common environment for Java programming these days is using
\href{https://maven.apache.org/index.html}{Maven}. The Maven tool, \texttt{mvn}, needs to be downloaded and installed,
which can be a project in itself. Then you need to build a Maven
environment, using an incantation like:

\begin{verbatim}
$ mvn -B archetype:generate \
    -DarchetypeGroupId=org.apache.maven.archetypes \
    -DgroupId=com.mycompany.app \
    -DartifactId=my-app

\end{verbatim}
\begin{verbatim}
$ mvn -B archetype:generate \
      -DarchetypeGroupId=org.apache.maven.archetypes \
      -DgroupId=com.salveteomnis.hello \
      -DartifactId=hello
\end{verbatim}

After which you get:
\begin{verbatim}

[INFO] Scanning for projects...
[INFO] 
[INFO] ------------------------------------------------------------------------
[INFO] Building Maven Stub Project (No POM) 1
[INFO] ------------------------------------------------------------------------
[INFO] 
[INFO] >>> maven-archetype-plugin:2.4:generate (default-cli) > generate-sources @ standalone-pom >>>
[INFO] 
[INFO] <<< maven-archetype-plugin:2.4:generate (default-cli) < generate-sources @ standalone-pom <<<
[INFO] 
[INFO] 
[INFO] --- maven-archetype-plugin:2.4:generate (default-cli) @ standalone-pom ---
[INFO] Generating project in Batch mode
[INFO] No archetype defined. Using maven-archetype-quickstart (org.apache.maven.archetypes:maven-archetype-quickstart:1.0)
[INFO] ----------------------------------------------------------------------------
[INFO] Using following parameters for creating project from Old (1.x) Archetype: maven-archetype-quickstart:1.0
[INFO] ----------------------------------------------------------------------------
[INFO] Parameter: basedir, Value: /Users/chasrmartin/Dropbox (Personal)/Tutoring/HowToProgram/YourVeryFirstProgram
[INFO] Parameter: package, Value: com.salveteomnis.hello
[INFO] Parameter: groupId, Value: com.salveteomnis.hello
[INFO] Parameter: artifactId, Value: hello
[INFO] Parameter: packageName, Value: com.salveteomnis.hello
[INFO] Parameter: version, Value: 1.0-SNAPSHOT
[INFO] project created from Old (1.x) Archetype in dir: /Users/chasrmartin/Dropbox (Personal)/Tutoring/HowToProgram/YourVeryFirstProgram/hello
[INFO] ------------------------------------------------------------------------
[INFO] BUILD SUCCESS
[INFO] ------------------------------------------------------------------------
[INFO] Total time: 5.839 s
[INFO] Finished at: 2017-11-14T10:52:00-07:00
[INFO] Final Memory: 17M/320M
[INFO] ------------------------------------------------------------------------
$
\end{verbatim}

This generates a "Hello, world" program for you, which is nice:

\begin{verbatim}
package com.salveteomnis.hello;

/**
 * Hello world!
 *
 */
public class App 
{
    public static void main( String[] args )
    {
        System.out.println( "Hello World!" );
    }
}

\end{verbatim}

It also generates mvn's own project description file, called the POM
file ("project object model", named in the days when to be a good
product it had to have the word "object" somewhere in it.)  That is an
XML-formatted file:

\begin{verbatim}
<project xmlns="http://maven.apache.org/POM/4.0.0" xmlns:xsi="http://www.w3.org/2001/XMLSchema-instance"
  xsi:schemaLocation="http://maven.apache.org/POM/4.0.0 http://maven.apache.org/maven-v4_0_0.xsd">
  <modelVersion>4.0.0</modelVersion>
  <groupId>com.salveteomnis.hello</groupId>
  <artifactId>hello</artifactId>
  <packaging>jar</packaging>
  <version>1.0-SNAPSHOT</version>
  <name>hello</name>
  <url>http://maven.apache.org</url>
  <dependencies>
    <dependency>
      <groupId>junit</groupId>
      <artifactId>junit</artifactId>
      <version>3.8.1</version>
      <scope>test</scope>
    </dependency>
  </dependencies>
</project>
\end{verbatim}

Which builds the whole directory tree Maven expects, all 20
directories and six files of it.

\begin{verbatim}
$ tree
.
├── pom.xml
├── src
│   ├── main
│   │   └── java
│   │       └── com
│   │           └── salveteomnis
│   │               └── hello
│   │                   └── App.java
│   └── test
│       └── java
│           └── com
│               └── salveteomnis
│                   └── hello
│                       └── AppTest.java
└── target
    ├── classes
    │   └── com
    │       └── salveteomnis
    │           └── hello
    │               └── App.class
    └── maven-status
        └── maven-compiler-plugin
            └── compile
                └── default-compile
                    ├── createdFiles.lst
                    └── inputFiles.lst

20 directories, 6 files
\end{verbatim}

After that, you can finally compile your program:

\begin{verbatim}
$ mvn compile
[INFO] Scanning for projects...
[INFO] 
[INFO] ------------------------------------------------------------------------
[INFO] Building hello 1.0-SNAPSHOT
[INFO] ------------------------------------------------------------------------
[INFO] 
[INFO] --- maven-resources-plugin:2.6:resources (default-resources) @ hello ---
[WARNING] Using platform encoding (UTF-8 actually) to copy filtered resources, i.e. build is platform dependent!
[INFO] skip non existing resourceDirectory /Users/chasrmartin/Dropbox (Personal)/Tutoring/HowToProgram/YourVeryFirstProgram/hello/src/main/resources
[INFO] 
[INFO] --- maven-compiler-plugin:3.1:compile (default-compile) @ hello ---
[INFO] Changes detected - recompiling the module!
[WARNING] File encoding has not been set, using platform encoding UTF-8, i.e. build is platform dependent!
[INFO] Compiling 1 source file to /Users/chasrmartin/Dropbox (Personal)/Tutoring/HowToProgram/YourVeryFirstProgram/hello/target/classes
[INFO] ------------------------------------------------------------------------
[INFO] BUILD SUCCESS
[INFO] ------------------------------------------------------------------------
[INFO] Total time: 2.566 s
[INFO] Finished at: 2017-11-14T11:04:01-07:00
[INFO] Final Memory: 14M/255M
[INFO] ------------------------------------------------------------------------
$ 

\end{verbatim}

If this sounds like I'm not a major fan of Maven, well, good. But at
the same time, Maven solves a lot of problems in Java programming,
like retrieving 3rd-party packages and making them available.  The
real point, though, is that you are going to have to solve all your
Maven issues to get started anyway, and you don't know you have until
you can run a "Hello, world" Maven project.

\section{Conclusion}
\label{sec:org2f9fe6b}

I do a lot of training and mentoring real beginners in
programming. Conssitently, one of the hardest problems for them is
what to do when you're looking at that first, empty project. But
anyone can set up a "Hello, world" program, and make it run, and that
first step means solving your first problems with any new project.
\end{document}